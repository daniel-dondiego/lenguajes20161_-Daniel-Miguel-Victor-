\documentclass[12pt]{article}
\usepackage[utf8]{inputenc}

\title{Lenguajes de Programación 2016-1\\Tarea 2}
\author{Ricardo Daniel Dondiego Pacheco \and  Miguel Ángel Mendoza Ponce \and Serrato Solano Victor Manuel}

\begin{document}
\maketitle
\section*{Problema I}
En teoría y laboratorio hemos visto el lenguaje FAE, que es un lenguaje
con expresiones aritméticas, funciones y aplicaciones de funciones.
¿Es FAE un lenguaje Turing-Completo?. Debes proveer una respuesta breve e inambigua, seguida de una justificación más extensa de tu respuesta.
Hint: Investiguen sobre el combinador Y
\textbf{Respuesta: }
Si lo es, ya que ésta gramática cuenta con estructuras de control y recursión.
Estructuras de control: es sencillo ver que la gramática las contiene ya que ...


\section*{Problema II}
¿Java es glotón o perezoso? Escribe un programa para determinar la respuesta a esta
pregunta. El mismo programa, ejecutado en cada uno de los dos regímenes, debe
producir resultados distintos. Puedes usar todas las características de Java
que gustes, pero debes mantener el programa relativamente corto:
\textbf{penalizaremos cualquier programa que consideremos excesivamente
  largo o confuso } (Hint: es posible resolver este problema con un programa
de unas cuantas docenas de lineas).
\\

\textbf{Respuesta: }
Java es glotón


\section*{Problema III}
En nuestro intérprete perezoso, identificamos 3 puntos en el lenguaje donde
necesitamos forzar la evaluación de las expresiones closures (invocando a la
función \texttt{strict}): la posición de la función de una aplicación,
la expresión de prueba de una condicional, y las primitivas aritméticas.
Doug Oord, un estudiante algo sedentario, sugiere que podemos reducir
la cantidad de código reemplazando todas las invocaciones de \texttt{strict}
por una sola. En el interprete visto en el capitulo 8 del libro de Shriram
elimino todas las instancias de \texttt{strict} y reemplazo
\begin{verbatim}
     [id (v) (lookup v env)]
\end{verbatim}

por

\begin{verbatim}
     [id (v) (strict (lookup v env))]
\end{verbatim}

El razonamiento de Doug es que el único momento en que el interprete regresa una
expresión closure es cuando busca un identificador en el ambiente. Si forzamos
esta evaluación, podemos estar seguros de que en ninguna otra parte del interprete
tendremos un closure de expresiones, y eliminando las otras invocaciones de
\texttt{strict} no causaremos ningún daño. Doug evita razonar en la otra dirección,
es decir si esto resultara o no en un interprete mas glotón de lo necesario.

Escribe un programa que produzca diferentes resultados sobre el interprete original
y el de Doug. Escribe el resultado bajo cada interprete e identifica claramente
cual interprete producirá cada resultado. Asume un lenguaje interpretado
con características aritméticas, funciones de primera clase, with, if0 y rec
(aunque algunas no se encuentren en nuestro interprete perezoso). Hint: Compara
este comportamiento contra el interprete perezoso que vimos en clase y no contra
el comportamiento de Haskell.

Si no puedes encontrar un programa como el que se pide, defiende tus razones
de por que no puede existir, luego considera el mismo lenguaje con cons, first
y rest añadidos.
\\

\textbf{Respuesta: } 
Se anexan dos archivos:
En ejercicio3-original.rkt se usa el intérprete original,en ejercicio3-doug.rkt se utiliza el intérprete de doug.

Ambos utilizan la misma expresión pero el intérprete original devuelve :
\begin{verbatim}
 > (exprV (num 10) (mtSub))
 \end{verbatim}
y el intérprete de doug devuelve:
\begin{verbatim}
 > (numV 10)
 \end{verbatim} 
Cuando los intérpretes llegan a lookup, el original devuelve el valor que le llega, en el de doug con stric t lo aplica hasta que no encuentra una exprV y ahí nos devuelve el resultado ya mencionado.


\section*{Problema IV}
Ningún lenguaje perezoso en la historia ha tenido operaciones de estado (tales
como la mutación de valores en cajas o asignación de valores a variables) ¿Por
que no?

La mejor respuesta a esta pregunta incluiría dos cosas: un pequeño programa (que
asume la evaluación perezosa) el cual usara estado y una breve explicación de cual
es el problema que ilustra la ejecución de dicho programa. Por favor usa la
noción original (sin cache) de perezosos sin cambio alguno. Si presentas un
ejemplo lo suficientemente ilustrativo (el cual no necesita ser muy largo), tu
explicación sera muy pequeña.
\end{document}